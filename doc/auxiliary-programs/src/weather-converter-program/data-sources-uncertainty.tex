\section{Data Sources/Uncertainty}\label{data-sourcesuncertainty}

More recent weather data source files have introduced the concept of data sources and uncertainty flags for many of the fields. The EnergyPlus weather format faithfully reproduces these fields as appropriate for the input source data types. By and large, most of the data sources and uncertainties have used the TMY2 established fields and values (See following table). As noted earlier, to enhance readability and reduce obfuscation, the EnergyPlus format for the data source and uncertainty flags collates them into one large field. Each data element still has its data source and uncertainty: it is positionally embodied depending on its place in the EPW data record.

% table 19
\begin{longtable}[c]{@{}ll@{}}
\caption{Key to Data Source and Uncertainty Flags \label{table:key-to-data-source-and-uncertainty-flags}} \tabularnewline
\toprule 
Data Flag & Flag Values \tabularnewline
\midrule
\endfirsthead

\caption[]{Key to Data Source and Uncertainty Flags} \tabularnewline
\toprule 
Data Flag & Flag Values \tabularnewline
\midrule
\endhead

Dry Bulb Temperature Data Source & A-F \tabularnewline
Dry Bulb Temperature Data Uncertainty & 0-9 \tabularnewline
Dew Point Temperature Data Source & A-F \tabularnewline
Dew Point Temperature Data Uncertainty & 0-9 \tabularnewline
Relative Humidity Data Source & A-F \tabularnewline
Relative Humidity Data Uncertainty & 0-9 \tabularnewline
Atmospheric Station Pressure Data Source & A-F \tabularnewline
Atmospheric Station Pressure Data Uncertainty & 0-9 \tabularnewline
Horizontal Infrared Radiation Data Source & A-H, ? \tabularnewline
Horizontal Infrared Radiation Data Uncertainty & 0-9 \tabularnewline
Global Horizontal Radiation Data Source & A-H, ? \tabularnewline
Global Horizontal Radiation Data Uncertainty & 0-9 \tabularnewline
Direct Normal Radiation Data Source & A-H, ? \tabularnewline
Direct Normal Radiation Data Uncertainty & 0-9 \tabularnewline
Diffuse Horizontal Radiation Data Source & A-H, ? \tabularnewline
Diffuse Horizontal Radiation Data Uncertainty & 0-9 \tabularnewline
Global Horizontal Illuminance Data Source & I, ? \tabularnewline
Global Horizontal Illuminance Data Uncertainty & 0-9 \tabularnewline
Direct Normal Illuminance Data Source & I, ? \tabularnewline
Direct Normal Illuminance Data Uncertainty & 0-9 \tabularnewline
Diffuse Horizontal Illuminance Data Source & I, ? \tabularnewline
Diffuse Horizontal Illuminance Data Uncertainty & 0-9 \tabularnewline
Zenith Luminance Data Source & I, ? \tabularnewline
Zenith Luminance Data Uncertainty & 0-9 \tabularnewline
Wind Direction Data Source & A-F \tabularnewline
Wind Direction Data Uncertainty & 0-9 \tabularnewline
Wind Speed Data Source & A-F \tabularnewline
Wind Speed Data Uncertainty & 0-9 \tabularnewline
Total Sky Cover Data Source & A-F \tabularnewline
Total Sky Cover Data Uncertainty & 0-9 \tabularnewline
Opaque Sky Cover Data Source & A-F \tabularnewline
Opaque Sky Cover Data Uncertainty & 0-9 \tabularnewline
Visibility Data Source & A-F, ? \tabularnewline
Visibility Data Uncertainty & 0-9 \tabularnewline
Ceiling Height Data Source & A-F, ? \tabularnewline
Ceiling Height Data Uncertainty & 0-9 \tabularnewline
Precipitable Water Data Source & A-F \tabularnewline
Precipitable Water Data Uncertainty & 0-9 \tabularnewline
Broadband Aerosol Optical Depth Data Source & A-F \tabularnewline
Broadband Aerosol Optical Depth Data Uncertainty & 0-9 \tabularnewline
Snow Depth Data Source & A-F, ? \tabularnewline
Snow Cover Data Uncertainty & 0-9 \tabularnewline
Days Since Last Snowfall Data Source & A-F, ? \tabularnewline
Days Since Last Snowfall Data Uncertainty & 0-9 \tabularnewline
\bottomrule
\end{longtable}

The definition of the solar radiation source flags and solar radiation uncertainty flags are shown in the following two tables:

% table 20
\begin{longtable}[c]{p{1.5in}p{4.5in}}
\caption{Solar Radiation and Illuminance Data Source Flag Codes \label{table:solar-radiation-and-illuminance-data-source}} \tabularnewline
\toprule 
Flag Code & Definition \tabularnewline
\midrule
\endfirsthead

\caption[]{Solar Radiation and Illuminance Data Source Flag Codes} \tabularnewline
\toprule 
Flag Code & Definition \tabularnewline
\midrule
\endhead

A & Post-1976 measured solar radiation data as received from NCDC or other sources \tabularnewline
B & Same as "A" except the global horizontal data underwent a calibration correction \tabularnewline
C & Pre-1976 measured global horizontal data (direct and diffuse were not measured before 1976), adjusted from solar to local time, usually with a calibration correction \tabularnewline
D & Data derived from the other two elements of solar radiation using the relationship, global = diffuse + direct ´ cosine (zenith) \tabularnewline
E & Modeled solar radiation data using inputs of observed sky cover (cloud amount) and aerosol optical depths derived from direct normal data collected at the same location \tabularnewline
F & Modeled solar radiation data using interpolated sky cover and aerosol optical depths derived from direct normal data collected at the same location \tabularnewline
G & Modeled solar radiation data using observed sky cover and aerosol optical depths estimated from geographical relationships \tabularnewline
H & Modeled solar radiation data using interpolated sky cover and estimated aerosol optical depths \tabularnewline
I & Modeled illuminance or luminance data derived from measured or modeled solar radiation data \tabularnewline
? & Source does not fit any of the above categories. Used for nighttime values and missing data \tabularnewline
\bottomrule
\end{longtable}

% table 21
\begin{longtable}[c]{@{}ll@{}}
\caption{Solar Radiation and Illuminance Data Uncertainty Flag Codes \label{table:solar-radiation-and-illuminance-data}} \tabularnewline
\toprule 
Flag & Uncertainty Range (\%) \tabularnewline
\midrule
\endfirsthead

\caption[]{Solar Radiation and Illuminance Data Uncertainty Flag Codes} \tabularnewline
\toprule 
Flag & Uncertainty Range (\%) \tabularnewline
\midrule
\endhead

1 & Not used \tabularnewline
2 & 2 - 4 \tabularnewline
3 & 4 - 6 \tabularnewline
4 & 6 - 9 \tabularnewline
5 & 9 - 13 \tabularnewline
6 & 13 - 18 \tabularnewline
7 & 18 - 25 \tabularnewline
8 & 25 - 35 \tabularnewline
9 & 35 - 50 \tabularnewline
0 & Not applicable \tabularnewline
\bottomrule
\end{longtable}

Finally, the Meteorological data source and uncertainty flag/codes are shown in the following two tables:

% table 22
\begin{longtable}[c]{p{1.5in}p{4.5in}}
\caption{Meteorological Data Source Flag Codes \label{table:meteorological-data-source-flag-codes}} \tabularnewline
\toprule 
Flag & Definition \tabularnewline
\midrule
\endfirsthead

\caption[]{Meteorological Data Source Flag Codes} \tabularnewline
\toprule 
Flag & Definition \tabularnewline
\midrule
\endhead

A & Data as received from NCDC, converted to SI units \tabularnewline
B & Linearly interpolated \tabularnewline
C & Non-linearly interpolated to fill data gaps from 6 to 47 hours in length \tabularnewline
D & Not used \tabularnewline
E & Modeled or estimated, except: precipitable water, calculated from radiosonde data; dew point temperature calculated from dry bulb temperature and relative humidity; and relative humidity calculated from dry bulb temperature and dew point temperature \tabularnewline
F & Precipitable water, calculated from surface vapor pressure; aerosol optical depth, estimated from geographic correlation \tabularnewline
? & Source does not fit any of the above. Used mostly for missing data \tabularnewline
\bottomrule
\end{longtable}

% table 23
\begin{longtable}[c]{p{1.5in}p{4.5in}}
\caption{Meteorological Uncertainty Flag Codes \label{table:meteorological-uncertainty-flag-codes}} \tabularnewline
\toprule 
Flag & Definition \tabularnewline
\midrule
\endfirsthead

\caption[]{Meteorological Uncertainty Flag Codes} \tabularnewline
\toprule 
Flag & Definition \tabularnewline
\midrule
\endhead

1- 6 & Not used \tabularnewline
7 & Uncertainty consistent with NWS practices and the instrument or observation used to obtain the data \tabularnewline
8 & Greater uncertainty than 7 because values were interpolated or estimated \tabularnewline
9 & Greater uncertainty than 8 or unknown. \tabularnewline
0 & Not definable. \tabularnewline
\bottomrule
\end{longtable}
