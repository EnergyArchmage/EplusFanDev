\section{WeatherData}\label{weatherdata}

The E/E+ format is very flexible (as well as being ASCII and somewhat readable). In addition to the usual weather data (temperatures, solar radiation data), the format embodies other information from the location and weather data (e.g.~design conditions, calculated ground temperatures, typical and extreme weather periods). The EPW (weather data format)~ is described in \href{file:///E:/Docs4PDFs/AuxiliaryPrograms.pdf}{Auxiliary Programs} Document. Other details including statistical reports, backgrounds on data sources and formats, use of the Weather Converter program (used both for processing data and reporting) are also provided in the Auxiliary Programs document.

The web site for EnergyPlus (http://www.energyplus.gov) provides downloadable weather data for many sites throughout the world from several different formats. In addition, we are amenable to posting more weather data from users.
