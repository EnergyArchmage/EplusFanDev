\section{Example 9. Demand Management}\label{example-9.-demand-management}

\subsection{Problem Statement}\label{problem-statement-011}

Demand management refers to controlling a building to reduce the peak electrical power draws or otherwise improve the load profile from the perspective of the electric utility. Managing electricity demand is an important application for EMS. We should ask, Can we take the model from example 2 and use the EMS to add demand management?

\subsection{EMS Design Discussion}\label{ems-design-discussion-011}

Example 2 is a model of a large office building, but unfortunately the utility tariff is not a demand-based rate. Therefore, we change to a different set of utility rate input objects so the model has demand charges.

For this example, we assume that demand is managed by turning down the lights and increasing the cooling setpoint. The EMS calling point chosen is ``BeginTimestepBeforePredictor'' because it allows you to change the lighting power levels and temperature setpoints before you predict the zone loads.

To manage the demand, we first need to develop some targets based on some a priori idea of what level of demand should be considered ``high.''~ Therefore, we first run the model without demand management and note the simulation results for demand. There are many ways to obtain the demand results, but one method is to obtain them from the tabular report for Tariffs called ``Native Variables.''~ In that report, the row called PeakDemand is the demand used to calculate demand charges and is listed in kW. We will use these values to construct a target level of demand for each month by taking these results and multiplying by 0.85 in an effort to reduce demand by 15\%. For example, the demand for January was 1,154.01 kW, so we make our target level to be 0.85 * 1154.01 = 980.91 kW and the demand for August was 1,555.20 kW, so the target is 0.85 * 1555.20 = 1,321.92 kW.

To develop our Erl program, we separate the overall task into two parts:

1)~~~Determine the current state of demand management control.

2)~~~Set the controls based on that control state.

We then divide the Erl programs into two main programs and give them descriptive names:~ ``Determine\_Current\_Demand\_Manage\_State'';~ ``Dispatch\_Demand\_Changes\_By\_State.''

The Erl program to determine the control state determines the current status for the demand management controls. You can record and manage the control state by setting the value of a global variable called ``argDmndMngrState.''~ For this example, we develop four control states that represent four levels of demand management:

\begin{itemize}
\item
  Level 1 is assigned a value of 0.0 and represents no override to implement changes to demand-related controls.
\item
  Level 2 is assigned a value of 1.0 and represents moderately aggressive overrides for demand-related controls.
\item
  Level 3 is assigned a value of 2.0 and represents more aggressive override.
\item
  Level 4 is assigned a value of 3.0 and represents the most aggressive overrides.
\end{itemize}

We develop an algorithm for choosing the control state by assuming it should be a function of how close the current power level is to the target power level, the current direction for changes in power use, and the recent history of control state. The current demand is obtained by using a sensor that is based on the ``Total Electric Demand'' output variable. This current demand is compared to the target demand levels discussed as well as a ``level 1'' demand level that is set to be 90\% of the target. If the current demand is higher than the level 1 demand but lower than the target, the state will tend to be at level 1. If the current demand is higher than the target, the current state will be either level 2 or level 3 depending on the trend direction. However, we do not want the response to be too quick because it leads to too much bouncing between control states. Therefore, we also introduce some numerical damping with the behavior that once a control state is selected it should be selected for at least two timesteps before dropping down to a lower level. This damping is modeled with the help of a trend variable that records the control state over time so we can retrieve what the control state was during the past two timesteps.

Once the control state is determined, the Erl programs will use EMS actuators to override controls based on the state. The following table summarizes the control adjustments used in our example for each control state.

% table 9
\begin{longtable}[c]{p{1.5in}p{2.54in}p{1.94in}}
\caption{Example 9 Demand Management Adjustments by Control State \label{table:example-9-demand-management-adjustments-by}} \tabularnewline
\toprule 
Control State & Lighting Power Ahjustment Factor & Cooling Thermostat Offset \tabularnewline
\midrule
\endfirsthead

\caption[]{Example 9 Demand Management Adjustments by Control State} \tabularnewline
\toprule 
Control State & Lighting Power Ahjustment Factor & Cooling Thermostat Offset \tabularnewline
\midrule
\endhead

0 & None & none \tabularnewline
1 & 0.9 & -   0.8ºC \tabularnewline
2 & 0.8 & -   1.5ºC \tabularnewline
3 & 0.7 & -   2.0ºC \tabularnewline
\bottomrule
\end{longtable}

For control state level 0, the actuators are all set to Null so they stop overriding controls and return the model to normal operation.

To alter the lighting power density with EMS, you could use either a direct method that employs a Lights actuator or an indirect method that modifies the lighting schedule. For this example we use the direct method with EnergyManagementSystem:Actuator input objects that enable you to override the Electric Power Level for each zone's lights. We also set up internal variables to obtain the Lighting Power Design Level for each Lights object.~ Finally, we set up an EMS sensor to obtain the lighting schedule value to use in Erl programs. If the demand management control state is 1, 2, or 3, we use the following model to determine a new lighting power level:

Power = (Adjustment Factor) × (Lighting Power Design Level) × (Schedule Value)

There are also two ways to alter the cooling setpoint with EMS. To dynamically alter the cooling setpoints, we modify the schedule rather than directly actuating Zone Temperature Control actuators. Changing the schedule allows one actuator to override all the zones; the more direct approach would require actuators for each zone. (This can be used to advantage if different zones are to be managed differently.)~ The algorithm applies an offset depending on the control state. In the input file, the schedule for cooling setpoints is called CLGSETP\_SCH, so we set up an actuator for this Schedule Value. Because the algorithm is a simple offset from the original schedule, we need to keep track of the values in the original schedule. We cannot use the same schedule as an input to the algorithm because once an actuator overrides the schedule value it will corrupt the original schedule. This would be an example of a circular reference problem. Therefore, we make a copy of the cooling setpoint schedule, naming it CLGSETP\_SCH\_Copy, and use the copy in a EnergyManagementSystem:Sensor object to obtain the current scheduled value for the setpoint. When we override the CLGSETP\_SCH schedule, it will not corrupt the values from the CLGSTEP\_SCH\_Copy schedule used as input.

\subsection{EMS Input Objects}\label{ems-input-objects-011}

The main input objects that implement this example of demand management are listed below and are included in the example file called ``EMSDemandManager\_LargeOffice.idf.''~ The results indicate that demand management controls could reduce electricity costs by around \$40,000 or 10\%.

\begin{lstlisting}

EnergyManagementSystem:ProgramCallingManager,
    Demand Manager Demonstration,
    BeginTimestepBeforePredictor,
    Determine_Current_Demand_Manage_State,
    Dispatch_Demand_Controls_By_State;


  EnergyManagementSystem:Program,
    Determine_Current_Demand_Manage_State,
    Set localDemand = CurntFacilityElectDemand / 1000.0 ,
    Set CurrntTrend = @TrendDirection FacilityElectTrend 4,
    IF (Month = = 1) ,
      Set argTargetDemand = 0.85 * 1154.01,
      Set argCrntDmnd = localDemand,
      Set argTrendDirection = CurrntTrend,
    ELSEIF (Month = = 2),
      Set argTargetDemand = 0.85 * 1150.85 ,
      Set argCrntDmnd = localDemand,
      Set argTrendDirection = CurrntTrend,
    ELSEIF (Month = = 3),
      Set argTargetDemand = 0.85 * 1313.56 ,
      Set argCrntDmnd = localDemand,
      Set argTrendDirection = CurrntTrend,
    ELSEIF (Month = = 4),
      Set argTargetDemand = 0.85 * 1364.28,
      Set argCrntDmnd = localDemand,
      Set argTrendDirection = CurrntTrend,
    ELSEIF (Month = = 5),
      Set argTargetDemand = 0.85 * 1506.29  ,
      Set argCrntDmnd = localDemand,
      Set argTrendDirection = CurrntTrend,
    ELSEIF (Month = = 6),
      Set argTargetDemand = 0.85 * 1516.93  ,
      Set argCrntDmnd = localDemand,
      Set argTrendDirection = CurrntTrend,


    ELSEIF (Month = = 7),
      Set argTargetDemand = 0.85 * 1545.20  ,
      Set argCrntDmnd = localDemand,
      Set argTrendDirection = CurrntTrend,
    ELSEIF (Month = = 8),
      Set argTargetDemand = 0.85 * 1555.20  ,
      Set argCrntDmnd = localDemand,
      Set argTrendDirection = CurrntTrend,
    ELSEIF (Month = = 9),
      Set argTargetDemand = 0.85 * 1491.38  ,
      Set argCrntDmnd = localDemand,
      Set argTrendDirection = CurrntTrend,
    ELSEIF (Month = = 10),
      Set argTargetDemand = 0.85 * 1402.86  ,
      Set argCrntDmnd = localDemand,
      Set argTrendDirection = CurrntTrend,
    ELSEIF (Month = = 11),
      Set argTargetDemand = 0.85 * 1418.69  ,
      Set argCrntDmnd = localDemand,
      Set argTrendDirection = CurrntTrend,
    ELSEIF (Month = = 12),
      Set argTargetDemand = 0.85 * 1440.48  ,
      Set argCrntDmnd = localDemand,
      Set argTrendDirection = CurrntTrend,
    ENDIF,
    Run Find_Demand_State;




  EnergyManagementSystem:Subroutine,
    Find_Demand_State,
    !  argTargetDemand       Input kW level target
    !  argCrntDmnd           Input  kW level current
    !  argTrendDirection            Input   J/hour
    !  argDmndMngrState     Output  value code, 0.0 = no management,
    !                        1.0 = level 1 demand management
    !                     2.0 = level 2 demand management
    !                     3.0 = level 3 demand management
    Set DmndStateX1 = @TrendValue Demand_Mgr_State_Trend 1,
    Set DmndStateX2 = @TrendValue Demand_Mgr_State_Trend 2,
    Set Level1Demand = 0.9 * argTargetDemand,
    Set argCrntDmnd = argCrntDmnd,
    Set argTargetDemand = argTargetDemand,
    SET argDmndMngrState = DmndStateX1, ! initialize to last state then model changes
    IF (argCrntDmnd > Level1Demand) && (argCrntDmnd <argTargetDemand) && (argTrendDirection > 0.0),


      IF DmndStateX1 < = 1.0,
        SET argDmndMngrState = 1.0,
      ELSEIF (DmndStateX1 = = 2.0) && (DmndStateX2 < 2.0),
        SET argDmndMngrState = 2.0,  ! go at least two timesteps at 2.0
      ELSEIF (DmndStateX1 = = 3.0) && (DmndStateX2 = = 3.0),
        SET argDmndMngrState = 2.0,
      ELSEIF (DmndStateX1 = = 3.0) && (DmndStateX2 = = 2.0),
        SET argDmndMngrState = 3.0,  ! go at least two timesteps at 3.0
      ENDIF,


    ELSEIF (argCrntDmnd > argTargetDemand) && (argTrendDirection < 0.0),
      IF DmndStateX1 < = 2.0,
        SET argDmndMngrState = 2.0,
      ELSEIF (DmndStateX1 = = 3.0) && (DmndStateX2 = = 2.0) , ! go at least two timesteps at 3.0
        SET argDmndMngrState = 3.0,
      ELSEIF (DmndStateX1 = = 3.0) && (DmndStateX2 = = 3.0),
        SET argDmndMngrState = 2.0,
      ENDIF,


    ELSEIF (argCrntDmnd > argTargetDemand) && (argTrendDirection > = 0.0),
      Set argDmndMngrState = 3.0,
    ENDIF;




  EnergyManagementSystem:Program,
    Dispatch_Demand_Controls_By_State,
    IF     (argDmndMngrState = = 0.0),
      RUN Unset_Demand_Controls,
    ELSEIF (argDmndMngrState = = 1.0),
      RUN Set_Demand_Level1_Controls,
    ELSEIF (argDmndMngrState = = 2.0),
      Run Set_Demand_Level2_Controls,
    ELSEIF (argDmndMngrState = = 3.0),
      Run Set_Demand_Level3_Controls,
    ENDIF;




  EnergyManagementSystem:Subroutine,
    Unset_Demand_Controls,
    SET Set_Cooling_Setpoint_Sched    = Null,
    SET Set_Basement_Lights           = Null,
    SET Set_Core_bottom_Lights        = Null,
    SET Set_Core_mid_Lights           = Null,
    SET Set_Core_top_Lights           = Null,
    SET Set_Perimeter_bot_ZN_3_Lights = Null,
    SET Set_Perimeter_bot_ZN_2_Lights = Null,
    SET Set_Perimeter_bot_ZN_1_Lights = Null,
    SET Set_Perimeter_bot_ZN_4_Lights = Null,
    SET Set_Perimeter_mid_ZN_3_Lights = Null,
    SET Set_Perimeter_mid_ZN_2_Lights = Null,
    SET Set_Perimeter_mid_ZN_1_Lights = Null,
    SET Set_Perimeter_mid_ZN_4_Lights = Null,
    SET Set_Perimeter_top_ZN_3_Lights = Null,
    SET Set_Perimeter_top_ZN_2_Lights = Null,
    SET Set_Perimeter_top_ZN_1_Lights = Null,
    SET Set_Perimeter_top_ZN_4_Lights = Null;


  EnergyManagementSystem:Subroutine,
    Set_Demand_Level1_Controls,
    ! set lighting power to 90% of what it would otherwise be
    SET Set_Cooling_Setpoint_Sched    = Cooling_Setpoint_Sched + 0.8, ! add 0.8 deg C to cooling setpoint
    SET Set_Basement_Lights           = 0.90 * Basement_Lights * BLDG_LIGHT_SCH,
    SET Set_Core_bottom_Lights        = 0.90 * Core_bottom_Lights * BLDG_LIGHT_SCH,
    SET Set_Core_mid_Lights           = 0.90 * Core_mid_Lights * BLDG_LIGHT_SCH,
    SET Set_Core_top_Lights           = 0.90 * Core_top_Lights * BLDG_LIGHT_SCH,
    SET Set_Perimeter_bot_ZN_3_Lights = 0.90 * Perimeter_bot_ZN_3_Lights * BLDG_LIGHT_SCH,
    SET Set_Perimeter_bot_ZN_2_Lights = 0.90 * Perimeter_bot_ZN_2_Lights * BLDG_LIGHT_SCH,
    SET Set_Perimeter_bot_ZN_1_Lights = 0.90 * Perimeter_bot_ZN_1_Lights * BLDG_LIGHT_SCH,
    SET Set_Perimeter_bot_ZN_4_Lights = 0.90 * Perimeter_bot_ZN_4_Lights * BLDG_LIGHT_SCH,
    SET Set_Perimeter_mid_ZN_3_Lights = 0.90 * Perimeter_mid_ZN_3_Lights * BLDG_LIGHT_SCH,
    SET Set_Perimeter_mid_ZN_2_Lights = 0.90 * Perimeter_mid_ZN_2_Lights * BLDG_LIGHT_SCH,
    SET Set_Perimeter_mid_ZN_1_Lights = 0.90 * Perimeter_mid_ZN_1_Lights * BLDG_LIGHT_SCH,
    SET Set_Perimeter_mid_ZN_4_Lights = 0.90 * Perimeter_mid_ZN_4_Lights * BLDG_LIGHT_SCH,
    SET Set_Perimeter_top_ZN_3_Lights = 0.90 * Perimeter_top_ZN_3_Lights * BLDG_LIGHT_SCH,
    SET Set_Perimeter_top_ZN_2_Lights = 0.90 * Perimeter_top_ZN_2_Lights * BLDG_LIGHT_SCH,
    SET Set_Perimeter_top_ZN_1_Lights = 0.90 * Perimeter_top_ZN_1_Lights * BLDG_LIGHT_SCH,
    SET Set_Perimeter_top_ZN_4_Lights = 0.90 * Perimeter_top_ZN_4_Lights * BLDG_LIGHT_SCH;


  EnergyManagementSystem:Subroutine,
    Set_Demand_Level2_Controls,
    ! set lighting power to 80% of what it would otherwise be
    SET Set_Cooling_Setpoint_Sched    = Cooling_Setpoint_Sched + 1.5, ! add 1.5 deg C to cooling setpoint
    SET Set_Basement_Lights           = 0.80 * Basement_Lights * BLDG_LIGHT_SCH,
    SET Set_Core_bottom_Lights        = 0.80 * Core_bottom_Lights * BLDG_LIGHT_SCH,
    SET Set_Core_mid_Lights           = 0.80 * Core_mid_Lights * BLDG_LIGHT_SCH,
    SET Set_Core_top_Lights           = 0.80 * Core_top_Lights * BLDG_LIGHT_SCH,
    SET Set_Perimeter_bot_ZN_3_Lights = 0.80 * Perimeter_bot_ZN_3_Lights * BLDG_LIGHT_SCH,
    SET Set_Perimeter_bot_ZN_2_Lights = 0.80 * Perimeter_bot_ZN_2_Lights * BLDG_LIGHT_SCH,
    SET Set_Perimeter_bot_ZN_1_Lights = 0.80 * Perimeter_bot_ZN_1_Lights * BLDG_LIGHT_SCH,
    SET Set_Perimeter_bot_ZN_4_Lights = 0.80 * Perimeter_bot_ZN_4_Lights * BLDG_LIGHT_SCH,
    SET Set_Perimeter_mid_ZN_3_Lights = 0.80 * Perimeter_mid_ZN_3_Lights * BLDG_LIGHT_SCH,
    SET Set_Perimeter_mid_ZN_2_Lights = 0.80 * Perimeter_mid_ZN_2_Lights * BLDG_LIGHT_SCH,
    SET Set_Perimeter_mid_ZN_1_Lights = 0.80 * Perimeter_mid_ZN_1_Lights * BLDG_LIGHT_SCH,
    SET Set_Perimeter_mid_ZN_4_Lights = 0.80 * Perimeter_mid_ZN_4_Lights * BLDG_LIGHT_SCH,
    SET Set_Perimeter_top_ZN_3_Lights = 0.80 * Perimeter_top_ZN_3_Lights * BLDG_LIGHT_SCH,
    SET Set_Perimeter_top_ZN_2_Lights = 0.80 * Perimeter_top_ZN_2_Lights * BLDG_LIGHT_SCH,
    SET Set_Perimeter_top_ZN_1_Lights = 0.80 * Perimeter_top_ZN_1_Lights * BLDG_LIGHT_SCH,
    SET Set_Perimeter_top_ZN_4_Lights = 0.80 * Perimeter_top_ZN_4_Lights * BLDG_LIGHT_SCH;




  EnergyManagementSystem:Subroutine,
    Set_Demand_Level3_Controls,
    ! set lighting power to 70% of what it would otherwise be
    SET Set_Cooling_Setpoint_Sched    = Cooling_Setpoint_Sched + 2.0, ! add 2.0 deg C to cooling setpoint
    SET Set_Basement_Lights           = 0.70 * Basement_Lights * BLDG_LIGHT_SCH,
    SET Set_Core_bottom_Lights        = 0.70 * Core_bottom_Lights * BLDG_LIGHT_SCH,
    SET Set_Core_mid_Lights           = 0.70 * Core_mid_Lights * BLDG_LIGHT_SCH,
    SET Set_Core_top_Lights           = 0.70 * Core_top_Lights * BLDG_LIGHT_SCH,
    SET Set_Perimeter_bot_ZN_3_Lights = 0.70 * Perimeter_bot_ZN_3_Lights * BLDG_LIGHT_SCH,
    SET Set_Perimeter_bot_ZN_2_Lights = 0.70 * Perimeter_bot_ZN_2_Lights * BLDG_LIGHT_SCH,
    SET Set_Perimeter_bot_ZN_1_Lights = 0.70 * Perimeter_bot_ZN_1_Lights * BLDG_LIGHT_SCH,
    SET Set_Perimeter_bot_ZN_4_Lights = 0.70 * Perimeter_bot_ZN_4_Lights * BLDG_LIGHT_SCH,
    SET Set_Perimeter_mid_ZN_3_Lights = 0.70 * Perimeter_mid_ZN_3_Lights * BLDG_LIGHT_SCH,
    SET Set_Perimeter_mid_ZN_2_Lights = 0.70 * Perimeter_mid_ZN_2_Lights * BLDG_LIGHT_SCH,
    SET Set_Perimeter_mid_ZN_1_Lights = 0.70 * Perimeter_mid_ZN_1_Lights * BLDG_LIGHT_SCH,
    SET Set_Perimeter_mid_ZN_4_Lights = 0.70 * Perimeter_mid_ZN_4_Lights * BLDG_LIGHT_SCH,
    SET Set_Perimeter_top_ZN_3_Lights = 0.70 * Perimeter_top_ZN_3_Lights * BLDG_LIGHT_SCH,
    SET Set_Perimeter_top_ZN_2_Lights = 0.70 * Perimeter_top_ZN_2_Lights * BLDG_LIGHT_SCH,
    SET Set_Perimeter_top_ZN_1_Lights = 0.70 * Perimeter_top_ZN_1_Lights * BLDG_LIGHT_SCH,
    SET Set_Perimeter_top_ZN_4_Lights = 0.70 * Perimeter_top_ZN_4_Lights * BLDG_LIGHT_SCH;






  EnergyManagementSystem:GlobalVariable,argTargetDemand;
  EnergyManagementSystem:GlobalVariable,argCrntDmnd;
  EnergyManagementSystem:GlobalVariable,argTrendDirection;
  EnergyManagementSystem:GlobalVariable,argDmndMngrState;


  EnergyManagementSystem:Sensor,
    BLDG_LIGHT_SCH, !- Name
    BLDG_LIGHT_SCH, !- Output:Variable or Output:Meter Index Key Name
    Schedule Value; !- Output:Variable or Output:Meter Name




  EnergyManagementSystem:Sensor,
    CurntFacilityElectDemand,  !- Name
    Whole Building,        !- Output:Variable or Output:Meter Index Key Name
    Facility Total Electric Demand Power; !- Output:Variable Name




  EnergyManagementSystem:TrendVariable,
    Demand_Mgr_State_Trend , !- Name
    argDmndMngrState, !- EMS Variable Name
    48 ; !- Number of Timesteps to be Logged


  EnergyManagementSystem:TrendVariable,
    FacilityElectTrend , !- Name
    CurntFacilityElectDemand, !- EMS Variable Name
    144 ; !- Number of Timesteps to be Logged


  EnergyManagementSystem:Sensor,
    Cooling_Setpoint_Sched,  !- Name
    CLGSETP_SCH_Copy,        !- Output:Variable or Output:Meter Index Key Name
    Schedule Value; !- Output:Variable or Output:Meter Name


  EnergyManagementSystem:Actuator,
    Set_Cooling_Setpoint_Sched,  !- Name
    CLGSETP_SCH ,     !- Actuated Component Unique Name
    Schedule:Compact,   !- Actuated Component Type
    Schedule Value    ; !- Actuated Component Control Type


  EnergyManagementSystem:OutputVariable,
    Erl Cooling Setpoint [C],    !- Name
    Set_Cooling_Setpoint_Sched,  !- EMS Variable Name
    Averaged   ,                 !- Type of Data in Variable
    ZoneTimestep;                !- Update Frequency




  Output:Variable,
    *,
    Erl Cooling Setpoint,
    Timestep;


  EnergyManagementSystem:Actuator,
    Set_Basement_Lights,  !- Name
    Basement_Lights ,     !- Actuated Component Unique Name
    Lights,               !- Actuated Component Type
    Electric Power Level; !- Actuated Component Control Type


  EnergyManagementSystem:InternalVariable,
    Basement_Lights , !- Name
    Basement_Lights , !- Internal Data Index Key Name
    Lighting Power Design Level ; !- Internal Data Type


  EnergyManagementSystem:Actuator,
    Set_Core_bottom_Lights, !- Name
    Core_bottom_Lights ,  !- Actuated Component Unique Name
    Lights,               !- Actuated Component Type
    Electric Power Level; !- Actuated Component Control Type
  EnergyManagementSystem:InternalVariable,
    Core_bottom_Lights , !- Name
    Core_bottom_Lights , !- Internal Data Index Key Name
    Lighting Power Design Level ; !- Internal Data Type
  << Snipped remaining Lights Sensors and Actuators >>
\end{lstlisting}
