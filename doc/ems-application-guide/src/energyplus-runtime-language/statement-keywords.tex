\section{Statement Keywords}\label{statement-keywords}

Every programming language has instructions or commands that tell the processor what to do. Erl supports a few types of program statements. Each line of an Erl program begins with a statement keyword. The syntax for each line depends on the keyword that starts that line. Only those listed in Table 1 are allowed.

\begin{longtable}{@{}ll>{\raggedright}p{3in}@{}}

\caption{Programming Statements in ERL \label{table:programming-statements-in-erl}} \tabularnewline
\toprule
Keyword & Syntax & Statement Description \tabularnewline
\midrule
\endfirsthead

\caption[]{Programming Statements in ERL} \tabularnewline
\toprule
Keyword & Syntax & Statement Description \tabularnewline
\midrule
\endhead

RUN & RUN <program name> & Calls another Erl program or subroutine. Returns to the calling point when completed. Recursive calling is allowed. \tabularnewline
RETURN & RETURN, & Prematurely exits a subroutine or program and returns control to the caller. \tabularnewline
SET & SET <variable> = <expression>, & Assigns the right-hand side to the left-hand side. If <variable> has not been used before, it is dynamically declared (with local scope). \textbf{Note: <variables> should NOT start with numerics.} \tabularnewline
IF & IF <expression>, & Begins an ``IF block.'' Conditional decision. If <expression> evaluates to anything other than zero, the block of statements after the IF is executed. \tabularnewline
ELSEIF & ELSEIF <expression> & Conditional decision that follows a regular IF block of instructions. If <expression> evaluates to anything other than zero, the block of instructions after the ELSEIF is executed. \tabularnewline
ELSE & ELSE, & Conditional decision. Associated with an IF statement, the block of statements after the ELSE is executed if <expression> evaluates to zero for preceding IF and ELSEIF statements. \tabularnewline
ENDIF & ENDIF, & Terminates IF block (required). \tabularnewline
WHILE & WHILE <expression> & Begins a ``WHILE block.'' Conditional decision. If <expression> evaluates to anything other than zero, the block of statements after the WHILE is repeatedly executed. \tabularnewline
ENDWHILE & ENDWHILE, & Terminates WHILE block (required). \tabularnewline
\bottomrule
\end{longtable}

\subsection{Rules for IF blocks:}\label{rules-for-if-blocks}

\begin{itemize}
\item
  IF blocks can be nested, but only up to five deep.
\item
  ELSE is optional. If omitted, the IF block is terminated by ENDIF.
\end{itemize}

IF-ELSEIF-ELSE-ENDIF blocks are allowed. If there are many ELSEIF statements, the first in the list that evaluates to true (1.0) is applied and the execution jumps to the ENDIF for that IF block. If no IF or ELSEIF is true, the ELSE condition (if any) is applied. A single IF block currently has a limit of 199 ELSEIF statements plus one ELSE statement.

\subsection{Rules for WHILE blocks:}\label{rules-for-while-blocks}

\begin{itemize}
\item
  WHILE blocks cannot be nested.
\item
  A WHILE block must be terminated by an ENDWHILE
\item
  The block is repeated until the expression following the WHILE statement evaluates to 0.0 (false).
\item
  Once the WHILE loop is entered, there is a maximum limit on the number of times the loop can be repeated. The limit is set at one million repetitions to protect against the possibility of an Erl program entering an infinite loop if the WHILE loop's expression is malformed so as to never evaluate to 0.0 (false).
\end{itemize}

Erl programs are entered into the input data file (IDF) using the input objects called EnergyManagementSystem:Program and EnergyManagementSystem:Subroutine. These objects use individual fields to store the statements for an Erl program. As with most EnergyPlus objects, each field is separated by a comma and typically given a separate line of text for readability. In this case, each field can be considered a separate line of Erl program code. Every input field (line of code) must conform to the following rules:

\begin{itemize}
\item
  Every input field contains only one statement.
\item
  Every field begins with a statement keyword that identifies what that particular line of code is doing.
\item
  The syntax for each statement depends on the keyword.
\item
  All field content (keywords, variable names, etc.) is case insensitive.
\item
  A comma (or semicolon if it is the last field) marks the end of every statement.
\item
  The maximum length for a field is 100 characters. If you enter a longer field, it will be truncated to the first 100 characters. This can have subtle effects if the remaining portion forms a viable expression.
\item
  The ``!'' character is for comments.
\end{itemize}

REMEMBER, every line needs to end in a comma or, if it is the last in the program, a semicolon.
