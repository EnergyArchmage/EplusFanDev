\chapter{Output}\label{output}

EnergyPlus produces several output files as shown in the section on ``Running EnergyPlus''.~ This section will discuss the data contained in the ``standard'' output files (\textbf{eplusout.eso, eplusout.mtr}). They, too, have data dictionaries but unlike the input files, the output data dictionary is contained within the output file. Thus, the basic structure of the standard output file is:

\begin{lstlisting}
Data Dictionary Information
End of Data Dictionary
Data
...
Data
End of Data
\end{lstlisting}

As with the IDF structure, there are rules associated with the interpretation of the standard output data dictionary. These rules are summarized as follows:

\begin{itemize}
\item
  The first item on each line is an integer which represents the ``report code''. This ``report code'' will be listed in the data section where it will also be the first item on each line, identifying the data. Only 2 lines in the output file will not have an integer as the first item (``End of Data Dictionary'' and ``End of Data'' lines).
\item
  The second item on each line is also an integer. This integer corresponds to the number of items left on the dictionary line. Each string consists of a variable name and units in square brackets. Square brackets are required for all strings. If there are no units associated with a particular variable, then there are no characters between the brackets.
\end{itemize}

Six standard items appear at the start of every EnergyPlus Standard Output File Data Dictionary:

\begin{lstlisting}
Program Version,EnergyPlus <version number indicated>
1,5,Environment Title[],Latitude[degrees],Longitude[degrees],Time Zone[],Elevation[m]
2,6,Day of Simulation[],Month[],Day of Month[],DST Indicator[1 = yes 0 = no],Hour[],StartMinute[],EndMinute[],DayType
3,3,Cumulative Day of Simulation[],Month[],Day of Month[],DST Indicator[1 = yes 0 = no],DayType  ! When Daily Output variables Requested
4,2,Cumulative Days of Simulation[],Month[]  ! When Monthly Output variables Requested
5,1,Cumulative Days of Simulation[] ! When Run Period Output variables Requested
\end{lstlisting}

\begin{itemize}
\item
  Item 0 is the program version statement.
\item
  Item 1 is produced at the beginning of each new ``environment'' (design day, run period).
\item
  Item 2 is produced prior to any variable reported at the timestep or hourly intervals. Hourly intervals will be shown with a start minute of 0.0 and an end minute of 60.0. Timestep intervals will show the appropriate start and end minutes.
\item
  Item 3 is produced prior to any variable reported at the daily interval.
\item
  Item 4 is produced prior to any variable reported at the monthly interval.
\item
  Item 5 is produced prior to any variable reported at the end of the ``environment''.
\end{itemize}

Following these five standard lines will be the variables requested for reporting from the input file (ref. Output:Variable object). For example:

\begin{lstlisting}
6,1,Environment,Site Outdoor Air Drybulb Temperature [C] !Hourly
40,1,ZONE ONE,Zone Total Internal Latent Gain Energy [J] !Hourly
68,1,ZONE ONE,Zone Mean Radiant Temperature [C] !Hourly
69,1,ZONE ONE,Zone Mean Air Temperature [C] !Hourly
70,1,ZONE ONE,Zone Air Heat Balance Surface Convection Rate [W] !Hourly
71,1,ZONE ONE,Zone Air Heat Balance Air Energy Storage Rate [W] !Hourly
\end{lstlisting}

This example illustrates the non-consecutive nature of the ``report codes''. Internally, EnergyPlus counts each variable that \emph{could} be reported. This is the assigned ``report code''. However, the user may not request each possible variable for reporting. Note that, currently, the requested reporting frequency is shown as a comment (!) line in the standard output file.

The data is produced when the actual simulation is performed (after the warmup days unless the Output:Diagnostics requesting ReportDuringWarmup is used). Data output is simpler in format than the data dictionary lines. From the dictionary above:

\begin{lstlisting}
1,DENVER STAPLETON INTL ARPT ANN HTG 99% CONDNS DB,  39.74,-105.18,  -7.00,1829.00
2,1,12,21, 0, 1, 0.00,60.00,WinterDesignDay
6,-16.
40,0.0
68,-19.8183039170649
69,-19.8220899203323
70,-3.175272922406513E-002
71,-3.181520440307718E-002
\end{lstlisting}

This output file can be easily turned into a form that is read into commonly used spreadsheet programs where it can be further analyzed, graphed, etc.

More details including graphs are shown in the Output Details and Examples Document under the file \textbf{eplusout.eso}.

\section{Using ReadVarsESO}\label{using-readvarseso}

\subsection{Creating Charts and Spreadsheet files from Output Variables}\label{creating-charts-and-spreadsheet-files-from-output-variables}

The ReadVarsESO program is distributed with EnergyPlus as a simple approach to converting the standard output files (\textbf{eplusout.eso, eplusout.mtr}) into files that can be put directly into a spreadsheet program and then used to create graphs or do other statistical operations. ReadVarsESO can read the complex output files and sort the data into time-based form, it is a very quick application but does not have a lot of features.~ Note that all the \textbf{Output:Meter} and \textbf{Output:Meter:Cumulative} objects are included on the \textbf{eplusout.eso} file as well as the \textbf{eplusout.mtr} file.~ You can choose the \textbf{Output:Meter:MeterFileOnly} or \textbf{Output:Meter:Cumulative:MeterFileOnly} objects if you do not want a particular meter to show up on the \textbf{eplusout.eso} file. If you wish to see only the metered outputs, the \textbf{eplusout.mtr} file will typically be a lot smaller than the \textbf{eplusout.eso} file.

The ReadVarsESO program has a very simple set of inputs. By default, you will get all the variables listed in the Output:Variable (\textbf{eplusout.eso}) or Output:Meter (\textbf{eplusout.mtr}) objects into the appropriate output files. The outputs from ReadVarsESO are limited to 255 variables (Microsoft Excel™ limit). You can tailor how many variables to list by specifying variables for the ReadVarsESO runs.

You can override the 255 variable limit by specifying an argument on the command line (\textbf{EP-Launch} has a special option for this). You use ``unlimited'' or ``nolimit'' on the command line to get as many variables into your output file as desired. Again, this will be limited either by the number of variables in the \textbf{eplusout.eso} or \textbf{eplusout.mtr} files or the contents of the ``rvi'' file. If you want to use this option, you must include two arguments to the command line of the ReadVars program -- 1) the ``rvi'' file and 2) the ``unlimited'' argument.

% table 41
\begin{longtable}[c]{p{1.5in}p{4.5in}}
\caption{ReadVarsESO Command Line Options \label{table:readvarseso-command-line-options}} \tabularnewline
\toprule 
Option & Description \tabularnewline
\midrule
\endfirsthead

\caption[]{ReadVarsESO Command Line Options} \tabularnewline
\toprule 
Option & Description \tabularnewline
\midrule
\endhead

< filename > & To use any of these options, you must include a file name (“rvi” file) as the first argument. \tabularnewline
Unlimited (or Nolimit) & Variables of any number will be produced on the output file. \tabularnewline
Timestep & Only variables with reported frequency “timestep” (or detailed) will be produced on the output file. \tabularnewline
Hourly & Only variables with reported frequency “hourly”~ will be produced on the output file. \tabularnewline
Daily & Only variables with reported frequency “daily”~ will be produced on the output file. \tabularnewline
Monthly & Only variables with reported frequency “monthly”~ will be produced on the output file. \tabularnewline
Annual (or RunPeriod) & Only variables with reported frequency “runperiod”~ will be produced on the output file. \tabularnewline
\bottomrule
\end{longtable}

In addition, another argument can be used so that the output file is only one time frequency. (By default, all variables -- hourly, monthly, annual, etc. are mixed together in the output file). By using ``Timestep'' as an argument, you would get only the TimeStep reported variables. Using ``Monthly'', only the monthly variables. This is not automated in either \textbf{EP-Launch} or the \textbf{RunEPlus} batch file but can easily be accomplished.

The program is run automatically from the \textbf{EP-Launch} program or the \textbf{RunEPlus} batch file (both these methods for executing EnergyPlus are described in the GettingStarted document). These programs use \textbf{\textless{}filename\textgreater{}.rvi} for input to the ReadVarsESO program executed first after the EnergyPlus execution and \textbf{\textless{}filename\textgreater{}.mvi} for the second execution. Ostensibly, the .rvi file would apply to the eplusout.eso file and the .mvi would apply to the eplusout.mtr file, BUT the contents of the files actually specify the ``\textbf{input}'' reporting file and the ``\textbf{output}'' reorganized file. Typical contents of an .rvi file are:

% table 42
\begin{longtable}[c]{p{2.1in}p{3.9in}}
\caption{"RVI" file contents \label{table:rvi-file-contents}} \tabularnewline
\toprule 
.rvi line description & Actual .rvi File Contents \tabularnewline
\midrule
\endfirsthead

\caption[]{"RVI" file contents} \tabularnewline
\toprule 
.rvi line description & Actual .rvi File Contents \tabularnewline
\midrule
\endhead

Input File & eplusout.eso \tabularnewline
Output File & eplusout.csv \tabularnewline
Variable Name & Site Outdoor Drybulb Temperature \tabularnewline
Variable Name & Zone Air Temperature \tabularnewline
Variable Name & Zone Air Humidity Ratio \tabularnewline
Variable Name & Zone Air System Sensible Cooling Rate \tabularnewline
Variable Name & Zone Air System Sensible Heating Rate \tabularnewline
Variable Name & Zone Total Internal Latent Gain Rate \tabularnewline
Specific Variable Name & COOLING COIL AIR OUTLET NODE,System Node Temperature \tabularnewline
Specific Variable Name & AIR LOOP OUTLET NODE,System Node Temperature \tabularnewline
Specific Variable Name & AIR LOOP OUTLET NODE,System Node Humidity Ratio \tabularnewline
Specific Variable Name & Mixed Air Node,System Node Mass Flow Rate \tabularnewline
Specific Variable Name & Outdoor air Inlet Node,System Node Mass Flow Rate \tabularnewline
Variable Name & Humidifier Water Consumption Rate \tabularnewline
Variable Name & Humidifier Electric Power \tabularnewline
Variable Name & Zone Air Relative Humidity \tabularnewline
Variable Name & Zone Predicted Moisture Load Moisture Transfer Rate \tabularnewline
Termination Line ~(optional) & 0 \tabularnewline
\bottomrule
\end{longtable}

Note that the first two lines of the file are ``input file'' (where to read the output variable values from) and ``output file'' (where to put the reorganized data). If you have only those two lines in an ``rvi'' file, the program will use all the available variables on indicated input file to produce the output.

ReadVarsESO takes the input stream but recognizes the date/time information and appropriately places the required data onto the ``output file''. Following these lines are a list of variables to be culled from the ``input file'' and placed onto the output file. ``Variable Name'' will take all variables matching that description whereas ``Specific Variable Name'' will only match the full description of the variable. So, in the above example, ``Zone Air Temperature'' will report air temperatures for all the zones (but available at the HVAC timestep if you choose the ``detailed'' reporting frequency in your input file) but ``AIR LOOP OUTLET NODE'' and ``COOLING COIL AIR OUTLET NODE'' will be the only values reported for the ``System Node Temperature'' variable (the node temperature is available for all nodes used in the simulation). The termination line (0) is included to terminate the input to the ReadVarsESO program and begin the scanning.

The output from ReadVarsESO is in the form commonly called ``comma de-limited'' or ``comma separated variable''. This format can be read easily in spreadsheet programs, such as Excel™.

Note as described in the ``Input for Output'' above, only variables as listed on the \textbf{eplusout.rdd} file are available for reporting. If you request others, they will become ``Warning'' messages in the \textbf{eplusout.err} file.

\begin{lstlisting}
** Warning ** The following Output variables were requested but not generated
**   ~~~   ** because IDF did not contain these elements or misspelled variable name -- check .rdd file
************* Key = *, VarName = SYSTEM SENSIBLE COOLING RATE
\end{lstlisting}

The above message was generated from an IDF that requested reporting of the ``SYSTEM SENSIBLE COOLING RATE'' but that variable was not available from the components in the execution.

% table 43
\begin{longtable}[c]{p{1.5in}p{4.5in}}
\caption{Example ReadVarsESO command lines and results \label{table:example-readvarseso-command-lines-and-results}} \tabularnewline
\toprule 
Command Line & Description/Result \tabularnewline
\midrule
\endfirsthead

\caption[]{Example ReadVarsESO command lines and results} \tabularnewline
\toprule 
Command Line & Description/Result \tabularnewline
\midrule
\endhead

ReadVarsESO & Take eplusout.eso and produce an eplusout.csv file with all variables (up to 255) on it \tabularnewline
ReadVarsESO my.rvi & Use the contents of “my.rvi” to produce the appropriate output file (limited to 255 variables) \tabularnewline
ReadVarsESO my.rvi unlimited & Use the contents of “my.rvi” to produce the appropriate output file (no longer limited to 255 variables) \tabularnewline
ReadVarsESO my.rvi Monthly & Use the contents of “my.rvi” to produce the appropriate output file and only produce those variables reported for “monthly” frequency (up to 255 variables) \tabularnewline
ReadVarsESO my.rvi Daily unlimited & Use the contents of “my.rvi” to produce the appropriate output file and only produce those variables reported for “daily” frequency (no longer limited to 255 variables) \tabularnewline
\bottomrule
\end{longtable}

